% vim: indentexpr= nocindent spell

\title{CAVE Project \\
Heap Design}

\author{\large Felix Boucher \\
    \texttt{fboucher9@gmail.com}}
\documentclass[12pt]{article}
\usepackage{times}
\usepackage{listings}
\lstset{language=C++, basicstyle=\footnotesize\ttfamily}


\begin{document}

\maketitle

\begin{abstract}

Design notes for heap used by CAVE project.  Enumerate features of heap,
describe how to test heap.

\end{abstract}

\section{Features}

Enumerate features of heap design.

\begin{itemize}

\item \emph{Compatible} Interface is compatible with \verb|malloc| and
\verb|free|.  This make it easy to replace code that was using standard memory
management.

\item \emph{Global Instance} Memory management is required from every corner of
the project so it is not always possible to have access to a heap instance.
For that reason the heap instance is global.

\item \emph{Allocation Only} The simplest heap strategy is to allocate memory and to
never free it.  With an object that implements this simple strategy we may use
it to implement a more complex heap.

\item \emph{Thread Safety} Memory management may be called from any thread so
the heap state must be protected from race conditions.  Mutual exclusion
objects are used to protect the heap state.  For that reason the Mutual
exclusion object may not be allocated on the heap, to avoid recursion.

\item \emph{Small and Large} The design must be optimized to handle many
allocations of small size and few allocations of large size.  Some profiling
may be required to determine what is small or large.

\item \emph{Alignment} Small allocations are aligned to a multiple of 8 bytes.
Large allocations are aligned to a multiple of 4096 bytes.

\item \emph{Side Data} The data used to track state of heap must be stored in a
separate area of memory.  This is to protect the state of the heap from
corruption and also to improve alignment of memory allocations.  The heap state
is required at allocation and free but not during use of data.  The heap state
need not remain in cpu caches after initialization phase.  Another reason is
for large allocations, by removing the heap header, the allocation may be
aligned on a page without wasting a page.

\item \emph{Fragmentation} The design must avoid fragmentation.  The heap
becomes fragmented when an unused region of memory is split to provide memory
to a new allocation.  After several split operations, we end up with holes and
none of the holes match the desired size.  The unused region must remain at the
original size so that it may be reused directly.

\end{itemize}

\section{Tests}

How to test the heap?

\begin{itemize}

\item \emph{Code coverage}  Design a set of tests that use each line of the
heap implementation.  In order to reach every line of the heap implementation
it may be necessary to simulate errors.  It may even be necessary to cause
segmentation faults during code coverage tests.

\item \emph{Leak detection}  Use leak detector to signal either a problem with
application or maybe a problem with the heap implementation.

\item \emph{Stress}  Random sequence of memory management calls and
verifications.  Create several threads and each thread must create and destroy
many allocations.  Each allocation must detect corruption before being freed.
At the end of the test there must not be any leaks are there must not be any
failures.  The heap must be useable before and after the test.  The frequency
of the actions must be high enough to stress the design.

\item \emph{Reuse}  Verify that heap reuses memory.  On first allocation, fill
memory with a unique pattern.  On next allocation the memory is reused if
memory is filled with the same unique pattern.

\item \emph{Fragmentation}  Find a way to detect fragmentation.  Fragmentation
causes the total amount of memory to increase due to an increasing number of
holes.  Select a set of allocation sizes.  Then do a random sequence of
allocations and deallocations.  At the end of the sequence allocate all and
verify total amount of memory.  The total amount of memory should be stable
when random sequences are repeated.

\end{itemize}

\end{document}

