\title{CAVE Project \\
Code Template}

\author{\large Felix Boucher \\
    \texttt{fboucher9@gmail.com}}
\documentclass[12pt]{article}
\usepackage{times}
\usepackage{listings}
\lstset{language=C++, basicstyle=\footnotesize\ttfamily}


\begin{document}

\maketitle

\begin{abstract}

Illustrate coding style of CAVE project via annotated code samples.  All of
the modules within the project have a common style selected with a specific
intention.  The annotations provide insight to the style choices.

\end{abstract}

\tableofcontents

\lstlistoflistings

\section{Introduction}

\section{Source Code}

\subsection{Interface}

\begin{lstlisting}[caption={cv\_demo.h}]
/* See LICENSE for license details */

#ifndef cv_demo_h_
#define cv_demo_h_

/*
 *  Module: cv_demo.h
 *
 *  Description: Sample header.
 */

#include <cv_demo_pred.h>

/*
 *  Structure: cv_demo
 *
 *  Description: Sample demo object.
 */

struct cv_demo {
    char const * pc_text0;
    /* -- */
    int i_number;
    int i_padding[1u];
};

#define cv_demo_initializer_ \
{ 0, 0, { 0 } }

void cv_demo_report( cv_demo const * p_this );

#endif /* #ifndef cv_demo_h_ */

/* end-of-file: cv_demo.h */
\end{lstlisting}

\subsubsection{Remarks}

\begin{itemize}

\item \emph{License}

\begin{lstlisting}
/* See LICENSE for license details */
\end{lstlisting}

    The first line is a reference to the license file found in root folder
    of the project.

\item \emph{Include guard}

\begin{lstlisting}
#ifndef cv_demo_h_
#define cv_demo_h_
...
#endif /* #ifndef cv_demo_h_ */
\end{lstlisting}

    Header files have an include guard.  Each file must have a unique name
    so that the include guard macro is also unique.  The include guard is
    a preprocessor macro so the name must end with an underscore.

\item \emph{Module comment block}

\begin{lstlisting}
/*
 *  Module: cv_demo.h
 *
 *  Description: Sample header.
 */
\end{lstlisting}

    After the include guard is the module comment block.  The comment block
    must minimally have a brief description of the module.  Leave an empty line
    before and after the comment block so that the editor recognizes the block
    as a paragraph.

\item \emph{Includes}

\begin{lstlisting}
#include <cv_demo_pred.h>
\end{lstlisting}

    After the module comment block and before any definitions are the include
    statements.  Try to include the minimal amount of headers.  Try to include
    predefinitions if possible instead of including an interface.  Try to
    avoid including any standard header files, or any OS-specific header
    files.

\item \emph{Structure}

    Definition of structure has a comment block with a brief description.
    Fields are grouped by type.  In this particular case the base alignment
    of the structure in pointer so the \texttt{pc\_text0} member may be
        alone.  The \texttt{i\_number} field requires padding since an
        \texttt{int} may be smaller than a pointer.

\item Functions may be grouped together after the structures or placed near
    the corresponding structure.  No need to comment the functions from
        the header, they will be commented from the implementation.

\item The last line is a comment that marks the end of the file.  This is
    useful when printing, or for viewing source code from an environment that
        does not show the file name.  For example, cat of file in a terminal.

\end{itemize}

\subsection{Implementation}

\begin{lstlisting}[caption={cv\_demo.c}]
/* See LICENSE for license details */

/*
 *  Module: cv_demo.c
 *
 *  Description Sample source file.
 */

#include <cv_demo.h>
#include <stdio.h>

/*
 *  Function: cv_demo_report
 *
 *  Description: Print report to standard output.
 */

void cv_demo_report( cv_demo const * p_this) {
    printf("%s:%d\n", p_this->pc_text0, p_this->i_number);
}

/* end-of-file: cv_demo.c */
\end{lstlisting}

\end{document}
