% vim: indentexpr= nocindent spell

\title{CAVE Project \\
Trace Design}

\author{\large Felix Boucher \\
    \texttt{fboucher9@gmail.com}}
\documentclass[12pt]{article}
\usepackage{times}
\usepackage{listings}
\lstset{language=C++, basicstyle=\footnotesize\ttfamily}


\begin{document}

\maketitle

\begin{abstract}

Design notes for trace used by CAVE project.  Enumerate features of trace,
describe how to test trace.

\end{abstract}

\section{Features}

Enumerate features of trace design.

\begin{itemize}

\item \emph{Log}

Information is optionally stored into log files.  Each application has its own
log file.  The file locking mechanism may be used to avoid more than one
process per log file.  The application is responsible of cleaning the log files
when they are no longer used.

\item \emph{Echo}

Information is optionally echoed to terminal or standard output file.  This is
useful for debugging interactive applications and also may be necessary when
all log files are locked by other processes.

\item \emph{Stack}

Information is optionally stored into a per-thread stack.  This information is
used to generate a stack backtrace.  This is useful for memory leak detection
and for debug assertions.

\item \emph{Profile}

Information is optionally accumulated into per-thread and per-process
statistics.  A profile report may be generated typically on exit of the
process.

\item \emph{Locking}

Design must avoid locking whenever possible.  Information must be stored
into per-thread storage and then flushes to per-process storage only when
necessary.

\item \emph{Arguments}

Each trace may optionally be extended with extra information.  Design must
not use variable argument lists also known as \verb|printf|.

\item \emph{Levels}

Information is filtered using log levels.  When number of levels is zero then
all traces are disabled.  The maximum number of levels is eight, same as
\verb|syslog|.

\end{itemize}

\section{Tests}

How to test the trace?

\end{document}

